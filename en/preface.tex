\chapter{Introduction}

%\section{Databases}
\par 
A \textit{database} is a collection of related data. By data, we mean known facts that can be computerized and that have implicit meaning as stated in Fundamentals of Database Systems \cite{ElmasryNavathe15}. We will consider that a database stores data relevant to an enterprise at a host that can be accessed via network.
Databases became deep-rooted in every business. 
Independently of the field that a company is focused on we can enumerate many reasons for considering a database storage deployment a good idea.
Let us show some examples of how databases are used through various business domains.

\TODO{examples}
\begin{itemize}
	\item E-commerce \\
	Every company that sells products online does should use a database. The bare minimum is to store offered products themselves and keeping track of purchases that were done by users.
	
	\item Social Media \\
	Every piece of information that has ever been published on social media, from photo through like or comment to friendship establishment, was stored somewhere and that place is a database. Usually the database that a social platform uses does its job in a background. Nevertheless there may occur events when the data storage reminds of its presence as it did on the most recent outage of Facebook. \cite{Facebook19}.
	
	\item Healthcare \\
	Easy accessibility of large amount of patient's data is a main reason to deploy a database at doctor's office or a healthcare organization.
	\TODO{https://blogs.wsj.com/cio/2013/12/03/healthcare-govs-heart-beats-for-nosql/}
	And that is just a beginning of digitization of this industry.
	
	\item Finances \\
	
	
\end{itemize}
And the list goes on.
 \\
%\section{Database Design}

%\section{Database Models}
\par
A \textit{data model} is a description of data, data relationships, data semantics, and consistency constraints. \\
\par
A \textit{database model} is a particular type of a data model that determines in what manner can be data in database organized, stored, represented and accessed in a database.
\\
\\
There are multiple kinds of specific database models known.
According to Fundamentals of Database Systems \cite{SilberschatzKorthSudarshan10} the main categorization of database models nowadays, according to what do they describe and on what type of user are aimed, is following:
\begin{itemize}
	\item Conceptual Data Models (High-Level) \\
		Reproduces real world objects along with their relationships and should be close to how business end-users perceive them.
	
	\item Representational Data Models (Implementation) \\
		In the middle between the two other model types there are representational data models which on the one hand are comprehensible by end-users and on the other hand are not too abstract so that they can be used as documentation for an actual database implementation of the modeled data.
	
	\item Physical Level Data Models(Low-Level) \\
		 In contrast to conceptual models the physical ones are tied with how data are stored physically at storage media showing all specific internal details that may be overwhelming in the case that the reader is a computer specialist.
\end{itemize}



\TODO{move to the chapter about databases in comparison with the historical proposition}
In 1975 American National Standards Institute \cite{ANSIArchitecture75} introduced its database architecture consisting of three layers:
\begin{itemize}
	\item  External Level
	\item Conceptual Level
	\item Physical Level
\end{itemize}


%\section{Diagrams}

\TODO{Modeling tools}

%\section{Data Lineage}

\par
The process of development and deployment of a database consists of multiple stages. At the beginning there is a high level view of why the database is needed and what purpose will it serve. Hopefully in some time the result is that the data described in the initial step are stored physically at some server. 
This way the data can be accessed and processed. 
What we want to achieve in this work is to make use of the individual steps taken during the design process, and make operations on data as transparent and traceable as possible even for business users that don't have a technical background.

\section{Goals}

\begin{itemize}
	\item Develop a component that extracts metadata from database models that were created using SAP PowerDesigner 
	\item Develop a component that extracts metadata from database models that were created using ER/Studio  \\
	\TODO{complete the list}
\end{itemize}

\section{Overview of Chapters}

\TODO{Introduction to each of the following chapters}

